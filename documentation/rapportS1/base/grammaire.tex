Trouver une grammaire riche, complète et accessible à n'importe quel utilisateur est difficile.
C'est pourquoi, en réalité deux grammaires sont présentées.

\begin{itemize}
 \item La première est dite de 'haut-niveau'.
Elle permet de décrire la majorité du contenu du jeu dans un langage simple de compréhension.
 \item La seconde est dite de 'bas-niveau'.
Elle permet de manipuler chaque attribut et est beaucoup plus proche de l'implémentation finale que la première grammaire.
C'est par exemple elle qui permettra de définir le comportement de l'intelligence artificielle, chose très difficile à mettre en oeurvre à haut-niveau.
\end{itemize}

Le schéma de compilation se complexifie alors : un fichier décrivant le jeu dans la grammaire de haut-niveau est compilée afin de donner un fichier
respectant la grammaire de bas-niveau. A ce niveau là, l'utilisateur peut effectuer de nouveaux ajouts ou modifications. Le second compilateur
produit alors le script final du jeu en javascript.

