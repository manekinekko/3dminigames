Trouver une grammaire riche, complète et accessible à n'importe quel utilisateur est difficile.
C'est pourquoi, en réalité deux grammaires sont présentées.

\begin{itemize}
 \item La première est dite de 'haut-niveau'.
Elle permet de décrire la majorité du contenu du jeu dans un langage simple de compréhension.
 \item La seconde est dite de 'bas-niveau'.
Elle permet de manipuler chaque attribut et est beaucoup plus proche de l'implémentation finale que la première grammaire.
C'est par exemple elle qui permettra de définir le comportement de l'intelligence artificielle, chose très difficile à mettre en oeurvre à haut-niveau.
\end{itemize}

Le schéma de compilation se complexifie alors : un fichier décrivant le jeu dans la grammaire de haut-niveau est compilé afin de donner un fichier
respectant la grammaire de bas-niveau. A ce niveau là, l'utilisateur peut effectuer de nouveaux ajouts ou modifications. Le second compilateur
produit alors le script final du jeu en javascript.

\subsection{Grammaire haut-niveau}

La description du jeu se déroule en plusieurs phases.
Elle commence par les entités du jeu : personnage, ennemis, alliés et leurs attributs.
Viennent ensuite les définitions d'actions de base et des commandes.
Enfin, les règles du jeu sont spécifiées.

\subsubsection{Définition des entités du jeu et de leurs attributs}

Il s'agit de lister les différents personnages ou objets qui apparaissent dans le jeu.
Pour faciliter le travail de l'utilisateur, diverses classes sont déjà créées pour définir les entités.
La classe de base est la classe Object avec des attributs de base comme la position, l'orientation, la taille, etc.
De celle-ci héritent de nombreuses autres classes comme Character, Vehicle ou Weapon chacun ayant des nouveaux attributs spécifiques.
\note{Inclure le schéma des classes ?}

La déclaration d'un nouveau personnage se fait alors via un identificateur et le mot-clef 'is'.
Par exemple \code{Mario is Character} permet de déclarer Mario comme un personnage de classe Character, 
et peut donc posséder tous les attributs de cette classe. Ces derniers sont initialisés à des valeurs par défaut.
Pour modifier cette initialisation, par exemple pour changer l'attribut \code{lifeMax} de Mario, les mots clefs 'has' et 'at' sont utilisés : 
\code{Mario has lifeMax at 3}.
De même, il est nécessaire de pouvoir rajouter de nouveaux attributs à Mario.
La syntaxe est la même. Ainsi, \code{Mario has energie at 8} ajoute un attribut énergie à Mario, initialisé à 8.
Les nouveaux attributs doivent obligatoirement être initialisés grâce à 'at'.

De plus, s'il existe divers personnages qui possèdent aussi cette énergie, il est souhaitable de ne pas avoir à répéter cette ligne.
Ainsi, de nouveaux types peuvent être crées grâce au mot-clef 'type' :
\code{type Plombier is character. Plombier has energie at 8. Mario is Plombier. Luigi is Plombier.}
\note{a-t'on l'héritage multiple et si oui, comment ce la se passe en cas d'attributs du même nom ?}

Ensuite, en même temps que la déclaration de la classe d'une nouvelle identité, il est possible de la définir comme jouée par le joueur, ennemi, allié ou neutre.
(Si cela n'est pas précisé, la valeur est mise à neutre par défaut.
% \code{Mario is Plombier player. Luigi is Plombier ally. type Boss is character ennemy. Bowser is Boss.}
De plus, dans le cas où ce n'est pas un personnage joué, il peut être déclaré comme multiple.
Cela signifie qu'il est possible d'en générer plusieurs avec la seule commande 'generate'.
Par exemple \code{Koopa is character enemy multiple. generate 5 Koopa in zone}. Permet de créer 5 Koopa dans \code{zone} qui aura dû être définie
auparavant. Outre 'in', les mots-clefs 'on' permet de placer des objets juste au-dessus de la zone définie par le mot suivant, et le mot-clef 'at' génère
le premier objet à l'endroit indiqué et les autres proches de celui-ci, tout en évitant les collisions.
\note{A confirmer que ce soit ça pour multiple}

Enfin, il est possible de manipuler des listes de ces classes.
\code{type humain is character. type elfe is character : type nain is character. type hobbit is character. type magicien is character. gandalf is magicien ally.
CommunauteDeLAnneau is list of 1 elfe with 1 nain with 2 humain with gandalf with 4 hobbit.}

\subsubsection{Autres classes prédéfinies}

D'autres classes sont prédéfinies : une classe Game concernant l'ensemble général du jeu, une classe Camera contenant les informations sur les caméras :
certains comportements classiques sont prédéfinies comme un suivi à la première ou troisième personne, ou une caméra libre.

De plus, 2 clases permettent de gérer respectivement les compteurs et les ressources de type temporels.

\subsubsection{Définition de nouvelles actions et assignation des commandes}

Il est ensuite possible de définir de nouvelles actions via le mot-clef 'definition' et 'means'.
Par exemple, en reprenant l'exemple de Mario : \code{Mario is character player. definition suicide means mario dies.}

Les commandes sont ensuite définies après le mot-clef 'command'. Elles sont générées soit par l'appui d'une touche X du clavier ('key X') soit par 
une action Y sur la souris ('mouse Y').
\code{command mouse rClick for suicide} permet par exemple de causer l'action nommée suicide lorsque le joueur clique sur le bouton droit de la souris.

De plus, de nombreuses actions sont prédéfinies pour les personnes du jeu comme jump ou move left ou concernant le jeu en général comme pause, save ou gameover.
\code{command mario is key space for jump, key Z for move forward, key Y for move backward, key S for move left, key D for move right.}

Il est possible de désactiver (respectivement d'activer) certaines touches du clavier ou action de la souris au cours du jeu.
Pour cela, il faut utiliser le mot-clef 'desactive' (respectivement 'activate').
\code{desactive key Z} désactive les actions lors de l'appui sur la touche Z.
Il est également possible de désactiver toutes les commandes \code{desactive commands} ou toutes les commandes claviers \code{desactive key}.

\subsubsection{Déclaration des règles du jeu}

Les règles du jeu sont définies par le mot-clef 'rule' et 'then'.
Par exemple \code{rule mario dies then gameover}.

Il est également possible de manipuler les attributs des différentes entités. Prenons le cas où un ennemi a été défini \code{bowser is character ennemy}
et que si mario entre en collision avec lui , alors il perd un point de vie, réduit son score de 100 et provoque un saut.
\code{rule mario touches bowser then sub 1 for life of mario, sub 100 for score of game, mario jump}.

Outre 'sub' d'autres mots clefs existent pour les autres opérations arithmétiques élémentaires, ainsi qu'un 'assign' qui change directement
la valeur de l'attribut.

Certains concepts sont déjà définies : ainsi 'touches' est causée par une entrée en collision entre les deux entités, dies signifient l'arrivée de l'attribut
life à 0, kills se produit lorsque le premier personnage tue le second.
\note{j'ai un peu de mal à voir tout comment marche les declencheur et les conséquences dans la grammaire, du coup, je veux bien que 
quelqu'un nous explique tout ca, il doit manquer pas mal de choses là encore.}

\subsubsection{Exemple de mini-jeu décrit dans la grammaire 'haut-niveau'}

\note{Nous pensions éventuellement à un simple jeu à caméra fixe avec terrain prédéfini ou le personnage doit éviter certains ennemis
et prendre toutes les pièces avant d'atteindre un point d'arrivée, en fait genre un pacman sans bonus quoi :p, parce que mario on a le problème
des collisions par le haut ou pas, ainsi que la génération du terrain}

\subsection{Grammaire bas-niveau}

La grammaire bas-niveau est beaucoup plus proche de l'implémentation finale que la haut-niveau.
Ainsi, toutes les ressources doivent avoir un identifiant unique, les évènements qu'ils proviennent du joueur (clavier, souris) ou de la moficiation
de l'état d'un personnage sont gérés par des signaux.
La description d'un jeu est constitué de ressources, d'entités, de caméra, d'une boucle de rafraîchissement, d'un gestionnaire d'évènements.
Un moteur physique permettant de tester les collisions et contenant également les forces générales du jeu (gravitation, vent, etc.) est également disponible.

\subsubsection{Définition des ressources}

Une ressource est une variable du jeu qui pourra ensuite être attribuée à une entité.
Elle est identifiée par un nom unique.

\code{marioLife 1} permet de créer une ressource marioLife initialisée à 1.

Il existe un type spécial de ressource : les ressources temporelles.
Celles-ci sont mises à jour via la boucle de rafraîchissement générale du jeu.
Pour la définir, cette fois 2 valeurs sont données suites à la ressource : le pas du timer, et sont initialisation :
\code{timeStarBonus 10000 0} permet de créer une ressource de nom 'timerStarBonus' (pouvant correspondre à la durée du bonus donné par une étoile)
avec un pas de 10 secondes, et initialisée à 0 milisecondes. 

Il est également possible de faire des énumérations de ressources.
\note{il faudrait un joli exemple}

\subsubsection{Définition des entités et caméras}

On considère que le terrain est représenté par exemple par une matrice de points dans l'espace surlaquelle est aplliquée une texture.
La matrice pourra être remplacée selon le type de terrain pour un certain jeu. \note{Ca veut un peut rien dire ça, non ?}

L'ensemble des entités d'un jeu est alors constitué d'un terrain et d'un ensemble d'objet.
Chaque objet est identifié par un nom, ainsi que par deux fichiers créés par des outils de dessin d'objets 3D, un fichier .obj contenant l'ensemble
des points de l'objet, et un ficier .mat contenant la texture appliquée sur les faces de l'objet.
De plus chaque objet contient un ensemble de paramètre comme la vitesse, la position, etc. \note{vérifier le format de sortie de Sketchup}

Les caméras sont définies de manière similaire aux entités : chaque caméra est définie par un nom, une position et une orientation.

\note{comment on associe des ressources à une entité ?}

\subsubsection{Boucle de rafraîchissement et gestionnaire d'évènements}

Les divers évènements du jeu sont gérés via un système de signaux.

Le jeu dispose d'une boucle de rafraîchissement principale. Celle-ci génère un signal
signalUpdateCounter émettant un signal régulièrement.
De même, il permet de vérifier les touches enfoncées et actions de la souris à chaque tic du timer.

Enfin, chaque mise à jour de ressource nommée X émet un signal updateX.

Les actions lors de la réception d'un signal sont gérées par un gestionnaire d'évènements.
Ce dernier est défini par un ensemble de signaux et d'instructions.
Les instructions sont données comme dans un langage classique.
Elles sont composées des mises à jours de variable en utilisant les opérateurs arithmétiques classiques, les constantes, les autres ressources et des nombres
aléatoires.
Des instructions conditionnelles sont également disponibles, les booléens étant générés via des comparaisons d'expressions.

Les instructions peuvent également reprendre des concepts classiques de mini-jeux : pause, nouvelle partie, fin de partie, sauvegarde de la partie.
\note{insérer un petit exemple}

\subsubsection{Exemple de mini-jeu dans la grammaire bas-niveau}

\note{to do ...}