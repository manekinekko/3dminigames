Le but de ce projet est de créer un outil auteur de mini-jeux 3D destiné au Web.
En effet, les récentes arrivées de HTML 5 et d'outils 3D pour le Web comme WebGL permettent désormais l'affichage d'objets 3D dans les pages Web.
Cependant, les contenus 3D intégrés dans les pages Web ne sont actuellement que très peu interactifs.
Fort du succès des jeux flash, les contenus interactifs 3D comme les mini-jeux ont naturellement leur place sur le Web.
Or s'il existe de nombreux outils tel que Google SketchUp pour créer et manipuler les objets 3D en eux-mêmes,
il reste tout de même un effort important en ce qui concerne les interactions avec ceux-ci et en particulier la création de mini-jeux en 3D.

\subsection{Présentation générale}

La conception d'un outil de création de mini-jeux 3D pour le Web nécessite, d'une part, la description des objectifs, des règles, des interactions, 
du scénario du jeu qui permettront de créer un script et d'autre part des élément 3D constitutifs. Un jeu peut également contenir
une partie de gestion d'une base de données, par exemple dans le but de pouvoir sauvegarder une partie.

\begin{figure}[h]
 \framebox[\linewidth]{\parbox{\linewidth}{~ \vspace{7cm} Schéma à faire ~\\~\\}}
 \caption{Schéma général du projet}
 \label{fig:schemaprojet}
\end{figure}

Le schéma général du projet est donc présenté sur la Figure \ref{fig:schemaprojet}.
La partie gauche représente un fichier écrit par l'utilisateur contenant les règles du jeu qu'il souhaite crée.
Il s'agit d'un simple fichier texte respectant une grammaire de description.

La partie centrale correspond à un compilateur.

Il permet de convertir le fichier de description du jeu en un script javascript exécutable dans une page Web et permettant de jouer et 
interagir avec l'environnement.
Le lien avec la 3D et WebGL se fait via un éditeur 3D classique permettant de dessiner les objets.
Dans certains cas, le script du jeu peut avoir recours à une base de données.

Chacun de ces aspects est détaillé dans les parties suivantes.

\subsection{Description d'un jeu}

Il s'agit de créer un langage, à la fois suffisamment abstrait pour être utilisé par une personne quelconque, mais également suffisamment riche
pour pouvoir décrire un maximum de jeux.
Le langage est défini par une grammaire.
Cette dernière doit permettre de décrire à la fois :
\begin{itemize}
 \item les objectifs ;
 \item les règles ;
 \item le scénario, les niveaux, la logique de score ;
 \item les interactions avec le(s) joueur(s).
\end{itemize}


Le challenge est difficile car il existe de très nombreuses catégories de jeux : 
par exemple, un jeu de gestion n'a, à première vue, aucun point commun avec un jeu de volley ou un jeu de plateforme.

Il serait illusoire de vouloir décrire absolument tous les jeux à l'aide d'une seule et unique grammaire.
En effet, les jeux décrits par une grammaire sont forcément restreints.

Toutefois, de nombreuses similarités existent entre plusieurs jeux. Il s'agit donc de les exploiter afin de définir un langage général de description de jeux.

\subsection{Compilation vers un script javascript}

Une fois la grammaire créée, notre plateforme de création de mini-jeux 3D offrira la possiblité à l'utilisateur de définir son jeu via un fichier.
L'écriture du fichier pourra se faire à l'aide d'un éditeur spécial par exemple créé via eclipse.

Le fichier est ensuite lu et parsé à l'aide d'un compilateur qui pourra être écrit à l'aide d'outil tel que ANTLR.
En sortie, des fichiers javascript seront produits.

\subsection{Edition des objets 3D et utilisation d'une base de données}

La création et l'édition des objets 3D se fera à l'aide d'outils déjà très complets tel que Google SketchUp.

\note{PW : rajouter plus de détails, j'y connais pas grand chose alors j'ai du mal à dire quoique ce soit}

\subsection{Remarque}

\note{PW : Ceci est un copier coller de ce qui est attendu concernant la présentation du sujet : il manque donc beaucoup de trucs dans ce blabla ...}

Rapport de présentation du projet (5 pages max). Il présente de manière synthétique  le projet et ses enjeux. Il doit comporter les éléments suivants :
\begin{itemize}
 \item Caractéristiques
 \item Avantages
 \item Bénéfices
 \item Etat de l’art synthétique de l’existant (si ce point est applicable)
\end{itemize}