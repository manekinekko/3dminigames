Ce projet consiste à créer un outil auteur de mini-jeux 3D destinés au Web.
En effet, les récentes arrivées de HTML 5 et d'outils 3D comme WebGL permettent désormais l'affichage d'objets 3D dictement
intégrés dans les pages internets.
Cependant, les contenus 3D dans les pages Web ne sont actuellement que très peu interactifs.
Fort du succès des jeux flash, les contenus interactifs 3D comme les mini-jeux ont naturellement leur place sur le Web.
Or s'il existe de nombreux outils tel que Google SketchUp pour créer et manipuler les objets 3D en eux-mêmes,
il reste tout de même un effort important à faire en ce qui concerne les interactions avec ceux-ci et en particulier la création de mini-jeux en 3D.

\vspace{0.5cm}

La conception d'un outil de création de mini-jeux 3D pour le Web nécessite, d'une part la description des objectifs, des règles, des interactions, 
du scénario du jeu qui permettra de générer le code du jeu, et d'autre part les élément 3D constitutifs.
Il peut également être souhaitable de pouvoir sauvegarder une partie :
un jeu ne peut pas forcément se finir en quelques minutes, le fait de pouvoir reprendre une partie commencée un autre jour est alors nécessaire.

\begin{figure}[h]
 \framebox[\linewidth]{\parbox{\linewidth}{~ \vspace{3cm} Schéma à faire ~\\~\\}}
 \caption{Schéma général du projet}
 \label{fig:schemaprojet}
\end{figure}

Le schéma général du projet est présenté sur la Figure \ref{fig:schemaprojet}.
La partie gauche représente un fichier écrit par l'utilisateur contenant les règles du jeu qu'il souhaite créer respectant
une certaine syntaxe qu'il faut définir.

Il s'agit de créer un langage, à la fois suffisamment abstrait pour être accessible à n'importe quel utilisateur, mais également suffisamment riche
pour pouvoir décrire un maximum de jeux.
Le langage est défini par une grammaire.
Cette dernière doit permettre de décrire à la fois :
\begin{itemize}
 \item les objectifs ;
 \item les règles ;
 \item le scénario, les niveaux, la logique de score ;
 \item les interactions avec le(s) joueur(s).
\end{itemize}

Le challenge est difficile car il existe de très nombreuses catégories de jeux : 
par exemple, un jeu de gestion n'a, à première vue, aucun point commun avec un jeu de volley ou un jeu de plateforme.

Il serait illusoire de vouloir décrire absolument tous les jeux à l'aide d'une seule et unique grammaire.
En effet, les jeux décrits par une grammaire sont forcément restreints.

Toutefois, de nombreuses similarités existent entre plusieurs jeux. Il s'agit donc de les exploiter afin de définir un langage général de description de jeux.

Le fichier définissant le jeu via la grammaire se fera à l'aide d'un éditeur spécial, par exemple créé via eclipse.

\vspace{0.5cm}

La partie centrale du schéma correspond à un compilateur.
Il permet de convertir le fichier de description du jeu en un script javascript exécutable dans une page Web et permettant de jouer et 
interagir avec l'environnement.
Le langage javascript a été choisi pour la génération du jeu car il est actuellement le plus utilisé pour les interactions dans les pages Web.

L'affichage 3D seront gérées à l'aide de WebGL.
La création et l'édition des objets 3D se fera à l'aide d'outils déjà très complets tel que Google SketchUp.
\note{compléter un peu la partie 3D, dire ce qu'est WebGL tout ca peut être ?}

\vspace{0.5cm}

Dans un premier temps, des exemples classiques de mini-jeux seront présentés et analysés afin de mieux identifier
les différences et points communs entre différents types de jeux. Cette analyse mettra en évidence des concepts classiques
présents dans plusieurs jeux.
Dans une seconde partie, les langages proposés pour la description des jeux seront exposés en mettant en évidence leurs possibilités et leurs limites.
Enfin, les concepts récurrents vus dans l'analyse et présents dans les grammaires seront détaillés.