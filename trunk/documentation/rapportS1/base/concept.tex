Certains concepts se trouvent dans beaucoup de mini-jeux. Le but est ici de les définir de manière précise.

\subsection*{Définition générale d'un jeu}

\underline{Jeu} : 
Activité de loisir soumise à des règles et ayant des objectifs (à la différence d'un jouet). Il s'agit ici d'un contenu numérique
 disponible via un site internet.

\underline{Règles du jeu} : 
Ensemble de principes qui décrivent le fonctionnement du jeu et 
la façon dont les éléments qui le composent interagissent entre eux. 
Ces règles définissent également les conditions nécessaires à la victoire (objectifs).

\subsection*{Élément d'un jeu}

\underline{Environnement} : 
Univers du jeu. 
Plus précisément, il s'agit des décors qui le composent, des entités et des objets qui s'y trouvent ainsi que 
des règles physiques qui s'y appliquent comme la gravitation.

\underline{Terrain} : 
Socle de l'environnement.

\underline{Entité} : 
Élément animé présent dans l'environnement ou pouvant changer d'état. 
Elle dispose de caractéristiques et d'un panel d'actions prédéfinies qui lui sont propres.

\underline{Décor} : 
Élément inerte présent dans l'environnement.


\subsection*{Titre à trouver x)}

\underline{Ressource} : 
Elément virtuel d'un jeu exprimé sous forme numérique, pouvant servir de monnaie, de matière première, d'état, etc.

\underline{Inventaire} : 
Ensemble des objets dont dispose une entité et qu'elle transporte avec elle.

\underline{Vie} : 
Ressource particulière associée aux entités mortelles et aux objets destructibles du jeu.


\subsection*{Contrôle direct/indirect}

\underline{Contrôle direct} :
Capacité du joueur à interagir sur l’environnement, limitée à une seule entité à un instant t. 
Les actions dont il dispose sont celles dont l'entité dispose.

\underline{Contrôle indirect} :
Capacité du joueur à interagir sur l’environnement, plus complexe. 
Les actions dont il dispose sont définies à la création du jeu et ne se limite pas au contrôle et aux actions d'une unique entité.
Il faut souvent une sélection de ou des entités avant d'agir dessus (ex : sélection de plusieurs soldats dans un STR).

\subsection*{Environnement linéaire/non linéaire}

\underline{Environnement linéaire} : 
Environnement dans lequel le joueur doit suivre un cheminement prédéfini (ex : Tomb Raider).

\underline{Environnement non linéaire} : 
Environnement dans lequel le joueur est libre d'évoluer comme il le souhaite (ex : World of Warcraft). 
Il n'y a donc pas d'obligation à accomplir les objectifs dans un ordre précis.


\subsection*{Autre ... à faire et à classer}

\underline{Niveau} :

\underline{Temps} :

\underline{Score} :

\underline{Allié,ennemi,neutre} :

\underline{Entité réactive/autonome}

