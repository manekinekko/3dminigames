Certains concepts se retrouvent dans beaucoup de mini-jeux. Le but de cette partie est de les définir de manière précise.

\subsection*{Définition générale d'un jeu}

\underline{Jeu} : 
Activité de loisir soumise à des règles et ayant des objectifs. Il s'agit ici d'un contenu numérique
 disponible via un site internet.

\underline{Règles du jeu} : 
Ensemble de principes qui décrivent le fonctionnement du jeu et 
la façon dont les éléments qui le composent interagissent entre eux. 
Ces règles définissent également les conditions nécessaires à la victoire (objectifs).

\underline{Niveau} :
Sous-division d'un jeu. Elle permet de gérer la difficulté de jeu de manière progressive. Elle peut aussi être utilisée pour en définir la fin. 

\subsection*{Élément d'un jeu}

\underline{Environnement} : 
Univers du jeu. Il s'agit des décors qui le composent, des entités et des objets qui s'y trouvent ainsi que 
des règles physiques qui s'y appliquent comme la gravitation.

\underline{Terrain} : 
Socle de l'environnement.
\note{mettre les différents types}

\underline{Entité} : 
Elément présent dans l'environnement pouvant changer d'état. 
Elle dispose de caractéristiques et d'un panel d'actions prédéfinies qui lui sont propres.

\underline{Décor} : 
Elément inerte présent dans l'environnement.

\underline{Ressource} : 
Elément virtuel d'un jeu exprimé sous forme numérique, pouvant servir de monnaie, de matière première, d'état ou autres.

\subsection*{Groupes d'Entités}

\note{Je propose un vote à main levée pour enlever Personnage car il peut être un véhicule, un projectile, ou n'importe quel autre objet}
\underline{Personnage} :
Entité qui agit avec un comportement particulier. Un personnage peut être le joueur, un allié, un ennemi ou neutre.

\underline{Joueur} :
Le joueur est une entité contrôlée par l'utilisateur grâce au clavier ou à la souris.

\underline{Allié} :
L'allié est une entité contrôlée par l'IA. Elle possède un comportement visant à aider le joueurlors de sa progression dans le jeu.

\underline{Ennemi} :
L'ennemi est une entité contrôlée par l'IA. Elle possède un comportement visant à faire perdre le joueur.

\underline{Neutre} :
Le neutre est une entité contrôlée par l'IA. Elle possède un comportement précis qui ne tient pas compte des autres groupes.


\subsection*{Liste de Ressources}

\note{Surement incomplet}

\underline{Inventaire} : 
Ensemble des objets dont dispose une entité et qu'elle transporte.

\underline{Bonus/Malus} :
Objet modifiant l'état d'une entité (ou plusieurs) de façon positive/négative. L'effet peut-être déclenché par une Entité ou par l'action de ramasser l'objet.

\underline{Vie} : 
Ressource particulière associée aux entités mortelles et aux objets destructibles du jeu.

\underline{Temps} :
Ressource particulière qui se calcule par rapport à la boucle de rafraichissement et évolue de façon cyclique (+1 tous les 3 rafraichissements par exemple)

\underline{Score} :
Ressource particulière qui se calcule par rapport aux actions du joueur et leurs conséquences sur la partie.
En fin de partie, le score permet un classement de tous les utilisateurs ayant déjà joué auparavant.

\subsection*{Contrôle direct/indirect}

\underline{Contrôle direct} :
Capacité du joueur à interagir sur l’environnement, limitée à une seule entité à un instant donné. 
Les actions dont il dispose sont celles dont l'entité dispose.

\underline{Contrôle indirect} :
Capacité du joueur à interagir sur l’environnement, plus complexe. 
Les actions dont il dispose sont définies à la création du jeu et ne se limitent pas au contrôle et aux actions d'une unique entité.
Il faut souvent une sélection de ou des entités avant d'agir dessus (ex : sélection de plusieurs soldats dans un jeu de type stratégie en temps réel).

\subsection*{Environnement linéaire/non linéaire}

\underline{Environnement linéaire} : 
Environnement dans lequel le joueur doit suivre un cheminement prédéfini (ex : Tomb Raider).

\underline{Environnement non linéaire} : 
Environnement dans lequel le joueur est libre d'évoluer comme il le souhaite (ex : World of Warcraft). 
Il n'y a donc pas d'obligation à accomplir les objectifs dans un ordre précis.
