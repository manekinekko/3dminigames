Les mini-jeux ont des genres très variés comme les jeux de gestion, de plateforme, de course.
L'analyse des différences et des ressemblances entre ceux-ci est d'autant plus complexe.

\vspace{0.5cm}

Les difficultés d'implémentation des mini-jeux sont elles aussi différentes.
Huit exemples de jeux sont décrits par la suite.
Cela permet d'appuyer l'analyse du contenu d'un mini-jeu.
Ces jeux n'ont pas été choisi par hasard. Chacun possède un intérêt dans la description de ses règles et son implémentation. 


\begin{center}
\begin{tabular}{l|l}
 Jeu & Intérêt majeur \\
 \hline
 Pacman & Système de bonus \\
 1942 & Tirs \\
 Volley & Logique de score \\
 Course & Intelligence artificielle \\
 Mario & Niveaux \\
 Watch'N'Droid & Vague d'ennemis \\
 Billard & Collisions \\
 Gestion & Ressources \\
\end{tabular}
\end{center}

\vspace{0.5cm}

Tout d'abord, chaque jeu est présenté succinctement, illustré par la capture d'écran d'une implémentation effectuée.
Dans un second temps, une analyse de leur contenu sera effectuée, appuyée par un tableau comparatif et les diagrammes UML de ces jeux.

\clearpage

\subsection{Présentation des mini-jeux}


\titlegame{Pacman}
\begin{minipage}{9cm}
Ce jeu est une adaptation du jeu classique de Pacman. 
Le joueur contrôle le Tux via les touches du clavier. 
Son but est de manger toutes les pommes sur le terrain tout en évitant les Microsoft qui essayent de l’attraper. 
Pour arriver à cet objectif, le joueur dispose de 3 vies. Le Tux peut manger des pommes en or pour pouvoir détruire les Microsoft pendant
une courte durée et marquer des points supplémentaires. Concernant l’affichage, le joueur voit le nombre de vies restantes, 
son score ainsi que le temps durant lequel les Microsoft restent vulnérables. Le jeu est adaptable car il est facile d'initialiser 
la carte à partir d’images et d’un fichier JSON.
\end{minipage}
\hfill
\begin{minipage}{6cm}
 \includegraphics[width=\linewidth]{img/capturejeu_pacman}
\end{minipage}


\titlegame{1942}
\begin{minipage}{6cm}
\includegraphics[width=\linewidth]{img/capturejeu_1942}
\end{minipage}
\hfill
\begin{minipage}{9cm}
Dans ce shoot'em up (jeu d’action où le joueur fait face à une multitude d’ennemis), 
le joueur contrôle un vaisseau armé de deux canons pour détruire tous les véhicules adverses et ainsi gagner des points. 
Le joueur possède 3 vies et doit engranger le maximum de points. Lorsque le joueur perd une vie, il devient invulnérable durant une courte durée 
pour reprendre la main. Le vaisseau est contrôlé via le clavier. En ce qui concerne les ennemis, 
ils suivent des déplacements prédéfinis qui peuvent être paramétrés. 
Le joueur n’est pas obligé de tuer tous les ennemis mais le but est de faire le plus grand score.
\end{minipage}

\titlegame{Volley}
\begin{minipage}{9cm}
CowCow Volley Party est un mini jeu humoristique mettant en scène deux vaches jouant au volley.
On peut jouer à ce jeu en mode solo, contre une IA ou à deux joueurs. 
Le but est de remporter 2 sets, sachant que remporter un set revient à marquer 21 points. 
La vache dispose de 3 coups différents : passe courte, passe longue et un smash.
Les règles de ce jeu sont les mêmes que celles du volley classique.
La vache est contrôlée au clavier aussi bien en mode multijoueur qu’en mode solo.
\end{minipage}
\hfill
\begin{minipage}{6cm}
 \includegraphics[width=\linewidth]{img/capturejeu_volleycowcow}
\end{minipage}

\clearpage
\titlegame{Course}
\begin{minipage}{6cm}
 \includegraphics[width=\linewidth]{img/capturejeu_course}
\end{minipage}
\hfill
\begin{minipage}{9cm}
Ce jeu de course futuriste pour le web permet au joueur de se frotter à des intelligences artificielles pour faire le meilleur temps. 
Ce n’est pas un jeu de course classique, des bonus se trouvent sur le circuit :
un turbo, un bonus permettant d’augmenter le chrono des concurrents, un autre permettant l'échange de sa position avec celle d'un adversaire, etc. 
De plus, il existe différentes zones de circuit ayant des effets divers : inversion des commandes, réduction de vitesse, etc.
Le joueur peut choisir son véhicule parmi une sélection de 12 vaisseaux différents contrôlés au clavier. 
Ce jeu est adaptable : en effet, de nouveaux bonus et de nouveaux circuits peuvent être ajoutés facilement.
Il est aussi possible de mettre en place un système de lecture de fichier pour configurer les paramètres des différentes courses.
\end{minipage}

\titlegame{Mario}
\begin{minipage}{9cm}
Ce mini-jeu de plateforme reprend le principe de jeux comme Mario Bros.
Le personnage est représenté par un rectangle et contrôlé au clavier. Il peut se déplacer sur les côtés ou sauter.
Il doit avancer au maximum sans tomber dans les trous ni être touché par des ennemis. Ces derniers sont symbolisés par des triangles.
Le personnage peut faire perdre des points de vie aux ennemis en leur sautant dessus jusqu'à les tuer. 
S'il les touche sur les côtés, il meurt et la partie est terminée.
La caméra avance lorsque le joueur avance suffisamment mais elle ne lui permet pas de revenir en arrière.
Le terrain est généré aléatoirement et est infini.
Le but du jeu est d'obtenir le plus grand score. Le score diminue avec le temps et augmente lorsqu'un ennemi est tué ou que le personnage avance.
Il est facilement possible de changer le type de fonctionnement du jeu pour passer dans un mode où le but est de passer d'un niveau à un autre
avec des niveaux prédéfinis au préalable.
\end{minipage}
\hfill
\begin{minipage}{6cm}
 \includegraphics[width=\linewidth]{img/capturejeu_mario}
\end{minipage}


\titlegame{Watch'N'Droid}
\begin{minipage}{6cm}
 \includegraphics[width=\linewidth]{img/capturejeu_watchndroid}
\end{minipage}
\hfill
\begin{minipage}{9cm}
Watch’N’Droid est un mini-jeu du style Game \& Watch. 
Le joueur doit chasser ses ennemis avant qu'ils ne l’atteignent. 
Ils apparaissent un par un en bas de l’écran et montent pour l'atteindre à une certaine vitesse.
Cette dernière varie en fonction du niveau dans lequel le joueur se trouve. 
Le personnage est armé de deux marteaux pour détruire les ennemis et marquer des points. 
Lorsque l’ennemi atteint le joueur, ce dernier perd une vie.
Lorsqu’il ne lui reste que deux vies, des vies bonus apparaissent sur l’écran et le joueur peut les ramasser. 
Lorsque le nombre de vie arrive à zéro, le joueur a la possibilité d’enregistrer son score s'il fait partie
 des cinq meilleurs qui sont actuellement enregistrés.
\end{minipage}

\titlegame{Billard}
\begin{minipage}{9cm}
Ce mini-jeu de billard est un jeu multijoueurs en tour par tour. 
Il reprend les règles classiques du billard anglais : chaque joueur doit rentrer toutes les boules d’une couleur puis la noire. 
D’un point de vue gameplay, le joueur actif (désigné par une icone verte) contrôle la queue de billard avec la souris. Cette dernière pointe automatiquement vers la boule blanche. 
Il peut ainsi choisir l’angle avec lequel il compte frapper la boule blanche. En fonction du temps pendant lequel
 il laisse le bouton de la souris enfoncé, le tir est plus ou moins fort. 
Pour l’affichage, chaque joueur voit le nombre de boules rentrées ainsi que sa couleur. 
Enfin, le cadre bleu en bas de l’écran affiche la personne ayant gagné la partie.
\end{minipage}
\hfill
\begin{minipage}{6cm}
 \includegraphics[width=\linewidth]{img/capturejeu_billard}
\end{minipage}


\titlegame{Gestion}
\begin{minipage}{6cm}
 \includegraphics[width=\linewidth]{img/capturejeu_gestion2}
\end{minipage}
\hfill
\begin{minipage}{9cm}
Commissariat est un jeu de gestion du style Farmville. 
Le but est de maintenir et améliorer le commissariat en fonction des ressources disponibles.
Les ressources sont au nombre de quatre : le nombre de policiers, l’argent, l’indice IGPN et l’alcool. 
Il y a trois actions disponibles pour le joueur. Il peut envoyer des policiers en mission dans un quartier 
choisi pour ramener de l’argent ainsi qu’un prisonnier. Si un prisonnier est ramené, le joueur a la possibilité 
de libérer le prisonnier contre de l’argent ou de le tabasser (avec un fort risque de pénalité). 
Il peut aussi acheter des équipements ainsi que de l’alcool avec son argent. L’alcool est une ressource qui diminue constamment, 
le joueur doit veiller à toujours en avoir pour ne pas perdre. Il peut aussi perdre avec un indice d’IGPN trop élevé, cet indice 
monte avec toutes les mauvaises actions comme tabasser un prisonnier.
\end{minipage}

\vspace{0.8cm}

\subsection{Analyse des mini-jeux}

Ces mini-jeux ont été développés et leurs diagrammes UML sont disponibles en annexe.
De plus, ils permettent d'appuyer l'analyse de la description d'un mini-jeu.
Le tableau suivant récapitule les différents objectifs des jeux et leur style de terrain.

\vspace{0.5cm}
\noindent
\begin{center}
\begin{tabular}{|l|| c|c|c|c|c|}
\hline
 Jeu &  Niveaux & Fin de niveau & Victoire & Score & Terrain \\
\hline
 Pacman & Oui & survie + tout ramassé & tous niveaux & Oui & Grille \\
\hline
 1942 & Oui & survie + ligne d'arrivée & tous niveaux & Oui & Progression verticale \\
\hline
 Volley &  Non & score & score & Oui & Plateau \\
\hline
 Course & Oui & ligne d'arrivée & tous niveaux & Oui & Ruban \\
\hline
 Mario & Oui & survie + ligne d'arrivée & tous niveaux & Oui & Progression horizontale\\
\hline
 Watch'N'Droid & Oui & tout tué & tous niveaux & Oui  & Plateau et grille\\
\hline
 Billard & Non & toute boule rentrée & score & Oui & Plateau \\
\hline
 Gestion & Non & Non & Non & Oui & Grille\\
\hline
\end{tabular}
\end{center}



Le second tableau permet de comparer les aspects concernant le contrôle et les personnages.

\vspace{0.5cm}
\noindent
\begin{tabular}{|l|| c|c|c|c|}
\hline
 Jeu & Personnage & Contrôle & Autres actions & PNJ \\
\hline
 Pacman &  Oui & direct via clavier & Aucune &  Oui \\
\hline
 1942 & Oui & direct via clavier & Tir & Oui  \\
\hline
 Volley & Oui & direct via clavier  & Tir, Saut & Oui \\
\hline
 Course & Oui & direct via clavier & Utilisation bonus & Oui \\
\hline
 Mario & Oui & direct via clavier &  Saut & Oui \\
\hline
 Watch'N'Droid & Oui & direct via clavier & Frappe & Oui  \\
\hline 
 Billard &  Oui {\small (si queue=joueur)} & direct via clavier & Force & Oui  \\
\hline
 Gestion &  Oui mais inutile & clic, sélection et action & Clic pour actions & Non \\
\hline
\end{tabular}

\vspace{0.5cm}

Ces tableaux permettent de mieux identifier les différences et points communs entre les jeux proposés.
Dans le second tableau, PNJ signifie Personne Non Joueur, soit toute entité qui a un comportement autonome.

Par exemple, les objectifs au cours d'un niveau (ou pour un jeu sans niveau) se limitent souvent à atteindre une certaine zone
appelée ligne d'arrivée, ou éventuellement à remplir des conditions de temps ou de ressources. En effet, la vie et le score
peuvent être vus comme des ressources. Une condition sur celles-ci est nécessaire à la victoire.
En faisant des conjonctions et des disjonctions de ces différentes possibilités,
les conditions de victoire pour tous les jeux (sauf le jeu de gestion) peuvent être définies.

En revanche, si on observe la façon dont le monde est construit, il est très différent d'un jeu à l'autre.
Pour certains comme Pacman, il est défini par une grille, pour d'autres comme le jeu de course, il est représenté via un ruban.

Le tableau permet de mettre en évidence le jeu de gestion par rapport aux autres jeux.
En effet celui-ci ne dispose pas de niveau et de conditions de victoire.
De plus, le personnage n'est pas directement contrôlé par l'utilisateur pour lui faire faire des actions.
La grammaire proposée ne traitera pas le cas des jeux de gestions.
Il serait possible de définir plusieurs grammaires, chacune couvrant une catégorie de jeux afin de pouvoir en 
créer plusieurs types. Ainsi, une autre grammaire pourrait spécialement être créée pour les jeux de gestion.

La comparaison se base donc sur les autres jeux, pour définir des concepts qui seront détaillés dans la partie suivante.

\vspace{0.5cm}

On remarque tout d'abord que tous les mini-jeux possèdent une boucle de rafraîchissement.
Cette dernière permet à la fois d'effectuer les différentes actions telles que le déplacement des personnages non joueurs et de gérer 
les différents compteurs du jeu (durée d'une course, d'un bonus).

Ensuite, dans les mini-jeux se retrouvent souvent la notion d'avatar (personnage que le joueur peut déplacer).
Différentes actions peuvent se greffer à cet avatar selon les jeux : souvent le déplacement, le saut, le tir.
Cet avatar a des caractéristiques que l'on retrouve dans plusieurs jeux comme la vie, d'autres plus rares comme une quantité de points de magie.
De même, dans de nombreux jeux se trouvent la notion d'ennemis. Ce sont des personnages non-joueurs dont le comportement est prédéfini et qui ont pour but d'empêcher le joueur de gagner.

De plus, la notion de ressources est très utile. Il s'agit d'une variable qui définit un état, par exemple, sur les entiers tel que le score.
Celle-ci peut être modifiée selon les évènements du jeu.

Enfin, dans tous les mini-jeux, le concept de collision est nécessaire. Celle-ci déclenche, la plupart du temps, 
de nombreux évènements comme la mort d'ennemis, de joueurs ou l'incapacité de passer un obstacle.