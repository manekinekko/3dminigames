\documentclass[a4paper,10pt]{article}

\usepackage[utf8]{inputenc}
\usepackage[T1]{fontenc}
\usepackage[french]{babel}
\AutoSpaceBeforeFDP

\usepackage{amsthm, amsmath,amsfonts, amssymb, latexsym}
\usepackage{color,graphicx,psfrag}
\usepackage{verbatim}
\usepackage{mathpazo}
\usepackage{url}
\usepackage{palatino}

\usepackage{fullpage}

\usepackage[svgnales]{xcolor}

\newcommand{\note}[1]{\textit{\textcolor{blue}{#1}}}

\newcommand{\titlegame}[1]{\subsubsection*{\begin{center}#1\end{center}}}

\usepackage{listings}

%% pour renommer la table des figures
\renewcommand{\listfigurename}{Liste des diagrammes UML}

%%%%%%%%%%%%%%%%%%%% Langage pour les jeux en haut-niveau
\definecolor{darkgreen}{rgb}{0,.5,0}

\lstdefinelanguage{GameGrammar}{
  morekeywords=[1]{type,is,has,duplicable,enemy,ally,neutral,player,game,at,in,on,list,of,with,means,disable,activate,definition,command,
                   rule,then,generate,efface,wait,endWait,random,ia,for,loop,once,during,if,else,endIf,play,not,
		   finished,started,paused,muted,paused,off,played,becomes,starts,ends}, % keywords
  morekeywords=[2]{Object,Character,Vehicle,SpaceCraft,Plane,Obstacle,Weapon,Projectile,Sword,Zone,Gound,Bonus,Checkpoint,Breakable,
		   Construction,Room,Ball,Teleporter}, % class
  morekeywords=[3]{mass,isFix,isTraversable,fov,type,active,name,description,life,
			lifeMax,lifeMin,nbOfLives,magic,magicMax,magicMin,level,attack,defense,jumpForce,
			maxJumpsInTheAir,money,class,race,acceleration,speed,maxSpeed,minSpeed,boost,
			maxBoost,nbMunitions,nbMunitionsMax,shootPower,damages,value,unit,objectname,
			attributName,volume,number,moveWithCamera,inventory,equipedObjects,entrances,
			exits,damageZone,collectors,typesCollectors,generators,typeGenerators,
			breakers,typesBreakers,teleportables,typesTeleportables,boostInterval,
			shootInterval,reloadTime
		  }, % attributs
  morekeywords=[4]{touches,dies,kills,killed,touched,jump,
		   keyboard,mouse,commands,
		   move,turn,left,right,forward,backward,accelerate,brake,
		   defeat,victory,
		   add,sub,assign,invert,
		   pause,mute,stop
		  }, % concept haut niveau
  morecomment=[l]{//},
  morecomment=[s]{/*}{*/},
  morestring=[b]',
}[keywords,comments,strings]

\lstdefinelanguage{Grammar}{
  morecomment=[l]{//},
  morecomment=[s]{/*}{*/},
  morestring=[b]',
}[keywords,comments,strings]

\lstset{
  language=GameGrammar,
  basicstyle=\scriptsize\ttfamily,
  commentstyle=\ttfamily\color{blue},
  keywordstyle=[1]\bfseries,
  keywordstyle=[2]\itshape,
  keywordstyle=[3]\itshape,
  keywordstyle=[4]\itshape\color{red},
  stringstyle=\color{darkgreen},
  linewidth=\linewidth,
  inputencoding=utf8,
  breaklines=true,
  showstringspaces=false,
  showspaces=false,
  showtabs=false
}

\newcommand{\code}[1]{\lstinline{#1}}

\lstnewenvironment{GameGrammar}[1][]{\lstset{language=GameGrammar}#1}{}



\title{Plateforme de création de "mini" jeux 3D sur le WEB \\ Rapport technique}

% \author{Berlon Antoine \and
% Bouzillard Jerôme \and
% Chegham Wassim \and
% Clergeau Thomas \and
% Faghihi Afshin \and
% Guichaoua Mathieu \and
% Israël Quentin \and
% Kien Emeric \and
% Lamine Gérald \and
% Le Corronc Thibault \and
% Le Galludec Benjamin \and
% Le Normand Erik \and
% Lubecki Aurélien \and
% Marginier David \and
% Sanvoisin Aurélien \and
% Tolba Mohamed Amine \and
% Weinzaepfel Philippe \and
% Zadith Ludovic}

\author{
A. Berlon \and
B. Bouzillard \and
W. Chegham \and
T. Clergeau \and
A. Faghihi \and
M. Guichaoua \and
Q. Israël \and
K. Kien \and
G. Lamine \and
T. Le Corron \and
B. Le Galludec \and
E. Le Normand \and
A. Lubecki \and
D. Marginier \and
A. Sanvoisin \and
A. Tolba Mohamed \and
P. Weinzaepfel \and
L. Zadith}

\date{\today}

\begin{document}
 
\maketitle

\begin{abstract}
 Ce document présente la partie analyse du projet de master 1 intitulé "Plateforme de création de mini-jeux 3D sur le WEB".
 Il consiste à concevoir et à développer un outil auteur pour créer des contenus 3D très interactifs dans les pages internet.
 En particulier, il s'agit de définir un langage permettant de générer le code de mini-jeux affichés en 3D.
 Le rapport est divisé en 4 parties : la partie \ref{sec:sujet} présente de manière plus précise le sujet.
 La seconde utilise l'exemple de différents mini-jeux pour appuyer l'analyse du contenu d'un mini-jeu.
 La partie \ref{sec:concept} définit les concepts récurrents dans l'univers des jeux.
 Enfin, la dernière partie présente les grammaires proposées.
\end{abstract}



\section{Présentation du sujet}
\label{sec:sujet}
Ce projet consiste à créer un outil auteur de mini-jeux 3D destinés au Web.
En effet, les récentes arrivées de HTML 5 et d'outils 3D comme WebGL permettent désormais l'affichage d'objets 3D dictement
intégrés dans les pages internets.
Cependant, les contenus 3D dans les pages Web ne sont actuellement que très peu interactifs.
Fort du succès des jeux flash, les contenus interactifs 3D comme les mini-jeux ont naturellement leur place sur le Web.
Or s'il existe de nombreux outils tel que Google SketchUp pour créer et manipuler les objets 3D en eux-mêmes,
il reste tout de même un effort important à faire en ce qui concerne les interactions avec ceux-ci et en particulier la création de mini-jeux en 3D.

\vspace{0.5cm}

La conception d'un outil de création de mini-jeux 3D pour le Web nécessite, d'une part la description des objectifs, des règles, des interactions, 
du scénario du jeu qui permettra de générer le code du jeu, et d'autre part les élément 3D constitutifs.
Il peut également être souhaitable de pouvoir sauvegarder une partie :
un jeu ne peut pas forcément se finir en quelques minutes, le fait de pouvoir reprendre une partie commencée un autre jour est alors nécessaire.

\begin{figure}[h]
 \centering
 \includegraphics[width=10cm]{img/schema_general}
 \label{fig:schemaprojet}
\end{figure}

Le schéma général du projet est présenté sur la Figure \ref{fig:schemaprojet}.
La partie gauche représente un fichier écrit par l'utilisateur contenant les règles du jeu qu'il souhaite créer respectant
une certaine syntaxe qu'il faut définir.

Il s'agit de créer un langage, à la fois suffisamment abstrait pour être accessible à n'importe quel utilisateur, mais également suffisamment riche
pour pouvoir décrire un maximum de jeux.
Le langage est défini par une grammaire.
Cette dernière doit permettre de décrire à la fois :
\begin{itemize}
 \item les objectifs ;
 \item les règles ;
 \item le scénario, les niveaux, la logique de score ;
 \item les interactions avec le(s) joueur(s).
\end{itemize}

Le challenge est difficile car il existe de très nombreuses catégories de jeux : 
par exemple, un jeu de gestion n'a, à première vue, aucun point commun avec un jeu de volley ou un jeu de plateforme.

Il serait illusoire de vouloir décrire absolument tous les jeux à l'aide d'une seule et unique grammaire.
En effet, les jeux décrits par une grammaire sont forcément restreints.

Toutefois, de nombreuses similarités existent entre plusieurs jeux. Il s'agit donc de les exploiter afin de définir un langage général de description de jeux.

Le fichier définissant le jeu via la grammaire se fera à l'aide d'un éditeur spécial, par exemple créé via eclipse.

\vspace{0.5cm}

La partie centrale du schéma correspond à un compilateur.
Il permet de convertir le fichier de description du jeu en un script javascript exécutable dans une page Web et permettant de jouer et 
interagir avec l'environnement.
Le langage javascript a été choisi pour la génération du jeu car il est actuellement le plus utilisé pour les interactions dans les pages Web.

La création et l'édition des objets 3D se fera à l'aide d'outils déjà très complets tel que Google SketchUp.
Leur affichage sera géré par WebGL.
\note{compléter un peu la partie 3D, dire ce qu'est WebGL tout ca peut être ?}

\vspace{0.5cm}

Dans un premier temps, des exemples classiques de mini-jeux seront présentés et analysés afin de mieux identifier
les différences et points communs entre différents types de jeux. Cette analyse mettra en évidence des concepts classiques
présents dans plusieurs jeux.
Dans une seconde partie, les langages proposés pour la description des jeux seront exposés en mettant en évidence leurs possibilités et leurs limites.
Enfin, les concepts récurrents vus dans l'analyse et présents dans les grammaires seront détaillés. 

\section{Les mini-jeux}
\label{sec:minijeux}
Le contenu des mini-jeux peut être très varié : jeu de rôle, jeu de gestion, jeu de plateforme, etc.
L'analyse des différences et des ressemblances entre ceux-ci, afin de définir ensuite une grammaire de description de jeux, est d'autant plus complexe.

Afin de nous confronter aux problèmes liés à l'implémentation de petits jeux vidéos et de faciliter l'analyse de la façon de décire un jeu, 
nous avons développé un échantillon de 8 types de mini-jeux en 2D dans le langage javascript.

Ces jeux n'ont pas été choisi par hasard : chacun possède un intérêt dans la description de ses règles et son implémentation.

\vspace{0.5cm}

\begin{tabular}{l|l}
 Jeu & Intérêt majeur \\
 \hline
 Pacman & ?? \\
 1942 & Vague d'ennemis \\
 Volley & Logique de score \\
 Course & Intelligence artificielle \\
 Mario & Niveaux \\
 Game\&Watch & ?? \\
 Billard & Collisions \\
 Jeu de Gestion & Ressources \\
\end{tabular}

\vspace{0.5cm}

Nous allons désormais présenter chacun de ces mini-jeux, pour ensuite analyser leurs objectifs, 
leurs règles et de façon plus générale leurs contenus afin d'en resortir différents aspects récurrents.

\subsection{Présentation des mini-jeux}

\note{c'est pour l'instant un bête copier coller de chacun, les fautes n'ont pas été corrigés.
On retravaillera cette sous-partie pour homogénéiser les différentes descriptions}

\subsubsection{Pacman}

Ce jeu est une adaptation du jeu classique de Pacman. 
Le joueur contrôle le TUX via les touches du clavier. 
Son but est de manger toutes les pommes sur le terrain tout en évitant les Microsoft qui essayent de l’attraper. 
Pour arriver a cet objectif, le joueur disposera de 3 vies. Le TUX pourra manger des pommes en or pour pouvoir détruire les Microsoft 
un petit temps et marquer quelques points en plus. En ce qui concerne l’affichage, le joueur verra le nombre de vie restante, 
son score ainsi que le temps ou les Microsoft restent vulnérables. Le Jeu est adaptable car il est facile d’initialiser 
la carte a partir d’images et d’un fichier Json.

\begin{figure}
 \includegraphics[width=\linewidth]{img/capturejeu_pacman}
 \caption{Capture du mini-jeu Pacman}
 \label{fig:game_pacman}
\end{figure}

\subsubsection{1942}

Dans ce shoot them up (jeu d’action où le joueur fait face à une multitude d’ennemis), 
le joueur contrôlera un vaisseau armé de deux canons pour détruire tous les véhicules adverses pour gagner des points. 
Le joueur aura 3 vies pour faire un maximum de point. Lorsque le joueur perd une vie, il devient invincible un petit moment pour reprendre la main ! 
Le vaisseau sera contrôlable via le clavier. En ce qui concerne les ennemis, ils suivent des déplacements prédéfinis qui peuvent être paramétrés. 
Le joueur n’est pas obligé de faire tuer tous les ennemis mais le but est quand même de faire le plus grand score !

\begin{figure}
 \includegraphics[width=\linewidth]{img/capturejeu_1942}
 \caption{Capture du mini-jeu 1942}
 \label{fig:game_1942}
\end{figure}

\subsubsection{Volley}


CowCow volley party  est un mini jeu humoristique mettant en scène deux vaches jouant au volley.
On peut jouer à ce jeu en mode solo avec une IA ou en multijoueur. 
Le but étant bien sur de gagner la partie en ayant 2 sets gagnants en premier. 
Pour cela votre vache dispose de 3 coups différents : passe courte et longue ainsi que le fameux smash !
Les règles de ce jeu sont les mêmes que les règles du volley classique. La vache sera contrôlée au clavier aussi bien  en mode multi qu’en mode solo.

\begin{figure}
 \includegraphics[width=\linewidth]{img/capturejeu_volleycowcow}
 \caption{Capture du mini-jeu volley}
 \label{fig:game_volley}
\end{figure}

\subsubsection{Course}

Ce jeu de course futuriste pour le web est un jeu pour un joueur qui se frottera à des intelligences artificielles pour faire le meilleur temps. 
Ce jeu de courses n’est pas un jeu de course classique, en effet sur le circuit vous trouverez des bonus 
(un turbo, un bonus permettant d’augmenter le temps d’un des autres joueurs, de changer leur positions, etc) 
et des zones de circuit différentes (inversion des commandes sur une zone, changement de vitesses, etc) 
ainsi que des objets fixes (turbo au sol,…). Vous pourrez choisir votre véhicule parmi une sélection de 12 vaisseaux différents. 
 En ce qui concerne les contrôles, le joueur se dirigera grâce à son clavier. Ce jeu est adaptable, il est, en effet, 
possible de créer facilement des circuits ainsi que des nouveaux bonus. Il serait aussi possible de mettre en place un système de 
lecture de fichier pour configurer les paramètres des différentes courses.

\begin{figure}
 \includegraphics[width=\linewidth]{img/capturejeu_course}
 \caption{Capture du mini-jeu de course}
 \label{fig:game_course}
\end{figure}

\subsubsection{Mario}

Ce mini-jeu de plateforme reprend le principe des jeux du style de mario bros.
Un personnage, représenté par un rectangle, peut se déplacer sur les côtés ou sauter.
Il doit avancer au maximum, sans être touché par des ennemis, représentés par des triangles et sans tomber dans les trous du terrain.
Le personnage peut faire perdre des points de vie aux ennemis jusqu'à les tuer en leur sautant dessus. S'il les touche sur les côté, 
il meurt et la partie est terminée.
La caméra avance lorsque le joueur avance suffisamment. Le joueur peut revenir à gauche jusqu'à la limite de l'écran.
Le terrain est généré aléatoirement et est infini.
Le but du jeu est d'obtenir le plus grand score. Le score diminue avec le temps et augmente lorsqu'un ennemi est tué ou que le personnage avance

\subsubsection{Game \& Watch}

Watch’N’Droid est un mini-jeu du style Game \& Watch. 
Le joueur doit chasser ses ennemis avant que ces derniers ne l’atteignent. 
Les ennemis apparaissent un par un en bas de l’écran et montent pour atteindre le joueur à une certaine vitesse. La vitesse des ennemis varie en fonction du niveau dans lequel le joueur se trouve. 
L’utilisateur est armé de deux marteaux pour détruire l’ennemi et donc marquer des points. 
Lorsque l’ennemi atteint le joueur se dernier perd une vie. 
Lorsqu’il ne lui reste que deux  vies, des vies bonus apparaissent sur l’écran et le joueur peut les ramasser. 
Lorsque le nombre de vie arrive a zéro, le joueur a la possibilité d’enregistrer son score si ce dernier fait parti des cinq meilleurs scores qui sont actuellement enregistrés.

\begin{figure}
 \includegraphics[width=\linewidth]{img/capturejeu_watchndroid}
 \caption{Capture du mini-jeu Game \& Watch}
 \label{fig:game_gamewatch}
\end{figure}

\subsubsection{Billard}

Ce mini-jeu de billard est un jeu multi-joueurs au tour par tour. 
Il reprend les règles classiques du billard anglais (chaque joueur doit rentrer toutes les boules d’une couleur puis la noire). 
D’un point de vue gameplay, le joueur contrôle la queue qui pointe automatiquement vers la boule blanche via sa souris. 
Le joueur peut ainsi choisir l’angle avec lequel il compte frapper la boule blanche et en fonction du pendant lequel le joueur va cliquer le tir va être plus ou moins fort. 
Pour l’affichage, chaque joueur voit combien de boules sont rentrés ainsi que sa couleur. Une icône verte apparait au niveau du joueur qui doit jouer. 
Le cadre bleu en bas de l’écran affiche la personne ayant gagné la partie !

\begin{figure}
 \includegraphics[width=\linewidth]{img/capturejeu_billard}
 \caption{Capture du mini-jeu billard}
 \label{fig:game_billard}
\end{figure}

\subsubsection{Gestion}

Le jeu commissariat est  un jeu de gestion du style Farmville. 
Vous aurez pour rôle de maintenir et améliorer votre commissariat en fonction des ressources disponibles.
 Les ressources sont au nombre de quatre : le nombre de policier, l’argent, l’indice IGPN et l’alcool ! 
Il y a trois actions disponibles pour le joueur. Il peut envoyer des policiers en missions dans un quartier 
choisis pour ramener de l’argent ainsi qu’un prisonnier.  Si un prisonnier est ramené, le joueur a la possibilité 
de libérer le prisonnier contre de l’argent ou de le tabasser (avec un fort risque de pénalité). 
Il peut aussi acheter des équipements ainsi que de l’alcool avec son argent. L’alcool est une ressource qui diminue constamment, 
le joueur doit ton veiller a toujours en avoir pour ne pas perdre. Il peut aussi perdre avec un indice d’IGPN trop élevé, cet indice 
monte avec toutes les mauvaises actions (tabasser un prisonnier, etc.).

\begin{figure}
 \includegraphics[width=\linewidth]{img/capturejeu_gestion2}
 \caption{Capture du mini-jeu de gestion}
 \label{fig:game_gestion}
\end{figure}

\subsection{Analyse des mini-jeux}

Suite au développement de ces mini-jeux, nous avons voulu analyser le contenu de ces différents jeux.
Pour cela, le tableau suivant récapitule différents aspects des jeux.

\note{inclure le tableau de google docs, en renvoyant peut être les colonnes pour mieux l'adapter à ce qu'on a fait dans la grammaire}

\vspace{0.5cm}

\begin{tabular}{|l|l}
\hline
 Jeu &   \\
\hline
 Pacman &  \\
\hline
 1942 &   \\
\hline
 Volley &  \\
\hline
 Course &  \\
\hline
 Mario &  \\
\hline
 Game\&Watch & \\
\hline
 Billard &  \\
\hline
 Jeu de Gestion & \\
\hline
\end{tabular}

\vspace{0.5cm}

Ce tableau permet de mieux identifier les différences et points communs entre les différents jeux.

On y voit par exemple que les objectifs au cours d'un niveau (ou pour un jeu sans niveau) revient souvent à atteindre une certaine zone
appelée ligne d'arrivée, ou éventuellement remplir des conditions de temps ou de ressources. En effet, à la fois la vie et le score
peuvent être vus comme une ressource : il s'agit d'un état, et une condition sur celui-ci est nécessaire à la victoire.
En faisant des conjonctions et des disjonctions de ces différentes possiblités, il est possible de définir 
les conditions de victoire pour tous les jeux (sauf le jeu de gestion).

En revanche, si on regarde comment le monde est construit, il est très différent d'un jeu à l'autre.
Pour certains comme Pacman, il est défini par une grille, pour d'autres comme le jeu de course, il est défini via un anneau.

Le tableau nous permet de voir assez clairement que le jeu de gestion est très différent des autres jeux.
Nous avons donc décidé de ne pas prendre en compte ce type de jeux.
Nous allons donc essayé de tirer des points communs entre les autres jeux, pour définir des concepts qui seront détaillés dans 
les parties suivantes.
Retirer de notre grammaire les jeux de gestion n'empêchent pas de couvrir un nombre important de jeux.
Il serait possible par exemple de définir plusieurs grammaires, chacune couvrant une catégorie de jeux afin de pouvoir
créer plusieurs types de jeux. Ainsi on pourrait créer une autre grammaire spécialement pour les jeux de gestion.
De la même façon, il est difficile de trouver des points communs entre les 7 mini-jeux restants, et un jeu de rôle où par exemple le joueur
doit réaliser plusieurs quetes.
Cependant, nous nous sommes concentrés pour le moment sur une unique grammaire décrivrant les 7 mini-jeux.

\note{Lister les concepts trouvés}


\section{Définition de concepts}
\label{sec:concept}
Certains concepts se trouvent dans beaucoup de mini-jeux. Le but est ici de les définir de manière précise.

\subsection*{Définition générale d'un jeu}

\underline{Jeu} : 
Activité de loisir soumise à des règles et ayant des objectifs (à la différence d'un jouet). Il s'agit ici d'un contenu numérique
 disponible via un site internet.

\underline{Règles du jeu} : 
Ensemble de principes qui décrivent le fonctionnement du jeu et 
la façon dont les éléments qui le composent interagissent entre eux. 
Ces règles définissent également les conditions nécessaires à la victoire (objectifs).

\subsection*{Element d'un jeu}

\underline{Environnement} : 
Univers du jeu. 
Plus précisément, il s'agit des décors qui le composent, des entités et des objets qui s'y trouvent ainsi que 
des règles physiques qui s'y appliquent comme la gravitation.

\underline{Terrain} : 
Socle de l'environnement.

\underline{Entité} : 
Elément animé présent dans l'environnement ou pouvant changer d'état. 
Elle dispose de caractéristiques et d'un panel d'actions prédéfinies qui lui sont propres.

\underline{Décor} : 
Elément inerte présent dans l'environnement.


\subsection*{Titre à trouver x)}

\underline{Ressource} : 
Elément virtuel d'un jeu exprimé sous forme numérique, pouvannt servir de monnaie, de matière première, d'état, etc.

\underline{Inventaire} : 
Ensemble des objets dont dispose une entité et qu'elle transporte avec elle.

\underline{Vie} : 
Ressource particulière associée aux entités mortelles et aux objets destructibles du jeu.


\subsection*{Contrôle direct/indirect}

\underline{Contrôle direct} :
Capacité du joueur à interagir sur l’environnement limitée à une seule entité à un instant t. 
Les actions dont il dispose sont celles dont l'entité dispose.

\underline{Contrôle indirect} :
Capacité du joueur à interagir sur l’environnement plus complexe. 
Les actions dont il dispose sont définies à la création du jeu et ne se limite pas au contrôle et aux actions d'une unique entité.

\subsection*{Environnement linéaire/non linéaie}

\underline{Environnement linéaire} : 
Environnement dans lequel le joueur doit suivre un cheminement prédéfini. (ex: Tomb Raider).

\underline{Environnement non linéaire} : 
Environnement dans lequel le joueur est libre d'évoluer comme il le souhaite. 
Il n'y a donc pas d'obligation à accomplir les objectifs dans un ordre précis.


\subsection*{Autre ... à faire et à classer}

\underline{Niveau} :

\underline{Score} :

\underline{Allié,ennemi,neutre} :

\underline{Entité réactive/autonome}



\section{Grammaires}
\label{sec:grammaire}
Trouver une grammaire riche, complète et accessible à n'importe quel utilisateur est difficile.
C'est pourquoi, en réalité deux grammaires sont présentées.

\begin{itemize}
 \item La première est dite de 'haut-niveau'.
Elle permet de décrire la majorité du contenu du jeu dans un langage simple de compréhension.
 \item La seconde est dite de 'bas-niveau'.
Elle permet de manipuler chaque attribut et est beaucoup plus proche de l'implémentation finale que la première grammaire.
C'est par exemple elle qui permettra de définir le comportement de l'intelligence artificielle, chose très difficile à mettre en oeurvre à haut-niveau.
\end{itemize}

Le schéma de compilation se complexifie alors : un fichier décrivant le jeu dans la grammaire de haut-niveau est compilée afin de donner un fichier
respectant la grammaire de bas-niveau. A ce niveau là, l'utilisateur peut effectuer de nouveaux ajouts ou modifications. Le second compilateur
produit alors le script final du jeu en javascript.



\section{Stratégie de Développement}
\label{sec:strategie}


\begin{figure}[h]
 \includegraphics[width=\textwidth]{strategie/diag_gantt}
\end{figure}

\begin{figure}[h]
 \includegraphics[width=\textwidth]{strategie/org_desc_tach}
\end{figure}

La première phase de développement se concentrera sur le compilateur du langage haut niveau qui ne nécessite pas d’apprentissage particulier 
puisque les bases du langage client (javascript) ont déjà été  étudiées lors de la phase d’analyse. 
En parallèle, sera effectuée une recherche générale sur l’utilisation de WebGL qui est un langage peu documenté car très récent, 
mais nécessaire à la suite du projet. Pour cette raison, une longue durée lui est consacrée.


WebGL incarne le principal risque du projet car nous ne savons pas le manipuler et nous ne connaissons pas ses limites.
 C’est une part d’inconnu qui pourrait potentiellement retarder le démarrage des tâches suivantes ainsi qu’affecter la bonne marche de celles 
liées au développement des compilateurs ; soit par obligation de modifier les grammaires établies pour répondre à des problèmes propres au WebGL 
soit par des difficultés d’implantation, causées par le manque de maturité du langage.

Une fois l’étude WebGL achevée, le développement du compilateur du langage bas niveau pourra commencer ainsi qu’un approfondissement de WebGL.


Lorsque les bases du compilateur du langage haut niveau seront établies, le développement de l’interface homme-machine pourra débuter.
 Un battement d’une semaine est prévu car il est fort probable que des modifications de ce compilateur aient lieu après une première finalisation.

La méthode agile répond aux besoins du projet, notamment à la souplesse qu’il demande. 
En conséquence, seront employés, le couple des méthodes Scrum et Extreme Programming. 

\clearpage
\appendix

\section{Grammaire haut-niveau}
\label{sec:hautniveau}
\subsection{La classe Object et ses filles}

Pour rappel, la classe object permet de définir toutes les entités du jeu.
Tout type défini par l'utilisateur hérite de cette classe ou de l'une de ces classes filles.
Elles sont présentées dans le schéma ci-dessous :

\begin{figure}[h]
 \centering
 \includegraphics[width=\textwidth]{img/objectclass}
\end{figure}

Les divers attributs 'name' pourront servir lors de l'affichage d'information.
\code{gandalf is Character. gandalf has name at 'Gandalf le Blanc'.}

Les classes Character, Vehicle et Plane/SpaceCraft ont des comportements différents au niveau des commandes.

Pour un objet de type Character, le déplacement se fait directement lors de l'appui sur une touche. 
Il est possible d'initialiser l'attribut 'acceleration' pour avoir une physique plus réaliste comme Sonic ou Mario.
Un attribut 'moveWithCamera' permet d'indiquer si le personnage se déplace en fonction de la caméra courante. 
Par exemple, indiquer au personnage 'move backward' sans cet attribut le fait reculer.
Avec 'moveWithCamera', le personnage avance vers la caméra.

Pour un ojet de type Vehicle, son déplacement se fait uniquement avec son accélération et le fait de tourner dans une direction
 se fait uniquement si le véhicule n'est pas à l'arrêt.

Enfin pour un objet de type Plane, SpaceCraft, sa masse est nulle s'il avance, et tourner dans une direction signifie
effectuer une rotation suivant l'axe que forme sa trajectoire.

Pour les Obstacles, ils héritent des attributs isFix et isTraversable de Object et sont mis respectivement à true et false.

Concernant les Projectiles, ils se déplacent dès qu'ils sont générés et peuvent avoir une zone de dégâts dans le cas d'un missile par exemle.

Une Zone correspond à un volume englobant invisible et traversable par tous.

La classe Ground peut avoir plusieurs types comme la neige, qui a donc un coefficient de frottement faible, ou l'eau qui est traversable.

Un Bonus possède une liste d'objets ou de types d'objets qui peuvent le ramasser. Dès qu'une entité de ce type entre en contact avec le bonus, celui-ci
disparaît et a un effet sur elle. 
Il contient aussi une liste d'objets ou de types d'objets qui peuvent le générer lorsqu'ils disparaîssent.

Le Checkpoint est une Zone particulière : si le joueur meurt, il réapparaît à cet endroit. 
Le checkpoint peut aussi servir dans un jeu de course pour obliger le joueur à passer à cet endroit (exemple : TrackMania).

Les objets de la classe Breakable correspondent à un Object à 2 états, représenté par 2 fichiers 3D différents. Les deux états correspondent
à une vie nulle ou une vie strictement positive. Un exemple concret est une fenêtre. 
Dès qu'une balle (Projectile ayant un attribut 'damages') la traverse, elle passe dans l'état cassée avec la représentation 3D associée.

Une Construction correspond à une vue extérieure, par exemple d'un bâtiment ou d'une maison.
Ce type d'objet est associé à Room pour des jeux de rôles où il n'est pas nécessaire de représenter une maison à sa taille réelle pour y entrer ensuite.
Une liste d'entrées correspond à des zones de téléporteurs associés le plus souvent aux sorties de Room.

Un objet de type Room reprend le principe de Construction mais les collisions sont gérées via des plans car un personnage à l'intérieur est considéré comme déjà en collision. 
Elle contient des sorties associées aux entrées d'une Construction ou d'une autre Room.

Un objet de type Ball a un volume englobant sphérique.

\subsection{Les autres classes}

Ci-dessous se trouvent les autres classes prédéfinies pour la grammaire haut-niveau.

\begin{figure}[h]
 \centering
 \includegraphics[width=\textwidth]{img/otherclass}
\end{figure}

La classe Game ne peut être instanciée qu'une seule fois.

Les ressources temporelles peuvent avoir plusieurs unités telles que les secondes, les millisecondes, les minutes ou les frames.

On peut définir la caméra selon un suivi à la première personne ('firstPerson') ou à la troisième personne ('thirdPerson') par rapport 
à l'objet qui aura été défini comme étant 'player'.
On peut aussi ne rien définir ou mettre 'free' pour une caméra libre.

Un Media a des mots-clefs dédiés comme play, mute on, mute off, pause ou stop.
On peut choisir entre un Media répété en boucle ('loop') ou lu une unique fois ('once').

\subsection{La grammaire}

L'intégralité de la gramaire haut-niveau est maintenant présentée.

\begin{lstlisting}[language=Grammar]

/*------------------------------------------------------------------
* PARSER RULES
*------------------------------------------------------------------*/
 
jeu :
  (infosJeu '.')?
  (nouveauType '.')*
  (init '.')+
  (definition '.')*
  (commande '.')*
  (reglesJeu '.')*
  (iaBasique '.')*;
 

///////////////////////////// ( informations about the game)  //////////////////////////////////

infosJeu :
  'Game' 'has' ('gravity' | 'score' /*| ...*/ ) 'at' (NUMBER | NUMBER NUMBER NUMBER)
  ;

 
//////////////////////////// ( Inheritance )  /////////////////////////////
 
nouveauType :
  'type' ident 'is' (ident | typeObjet) ('and' ident | typeObjet)* // to declare a new type
  ;            
  
// ident | typeObjet : if it is an ident, check that it is defined before by the user and that is an inherited Object.

 
//////////////////////////// ( Initializations )  /////////////////////////////

init :
  ident 'is' declarationObjet
  | accesClasse 'has' affectationObjet (',' affectationObjet)* // check the types and its attributes
  ;
 
declarationObjet :
  (ident | typeObjet3D) ('player' | interaction ('duplicable')? )?         // interaction is neutral by default
  | 'list' ('of' (operation)? (ident) ('with' (operation)? (ident))* )?  //operation if the object is duplicable
  | 'Camera' (('first' | 'third') 'person' | 'free')?
  | 'Media' ('loop' | 'once')? 						 // sound, music or video played in loop or once
  | 'in' ident 				                         // ident of a list to add an element
;           
 
interaction :
  'ally' | 'enemy' | 'neutral'
  ;
 
affectationObjet :
  ident ('at' (operation (uniteTps)? | ident) )?       //aggregation
  | attribut 'at' (operation | STRING | BOOL)          //life at 5, name at "Gandalf Le Gris"
  | typeCoordonnees 'at' coordonnees            //size at 20 30 40
  | attributListeOuObjet 'at' ident             //inventory at listeArmesJoueur
  | attributTps 'at' operation uniteTps         //
  ;
 
// has ident at ... : to declare a new attribute
// Attributes of predefined class have default initialized
// it is not necessary to initialize not used attribute
  
typeObjet :
  'Camera'
  | 'Media'
  | 'Counter'
  | 'Time'
  | typeObjet3D
  ;
 
// every predefined classes
typeObjet3D:
  'Object'                      // -> position(x,y,z), orientation(x,y,z), size(x,y,z)
  | 'Character'                 // -> life, lifeMax, magic, magicMax , level, experience, attack, defense
  | 'Vehicle'                   // -> acceleration, speedMax,
  | 'Plane' | 'SpaceCraft'
  | 'Obstacle'                  // a fixed entity, used for collisions
  | 'Weapon'                    // -> nbMunitions, nbMaxMunitions, intervalleTirs, timeRecharge
  | 'Sword'                     // -> damages, level
  | 'Projectile'                // -> vitesse, damages, level(pourquoi pas)
  | 'Zone'                      // an invisible and traversable entity
  | 'Ground'                    // -> type of ground (water, snow ...)
  | 'Bonus'                     // an object which disappears when something touches it-> valeur(entier), nomObjet(type),listeObjets 
  | 'CheckPoint'
  | 'Breakable'
  | 'Construction'
  | 'Room'
  | 'Ball'
  | 'Teleporter'
  ;
 
// every attributes of predefined classes
attribut : 
  'mass'                  // attributes of object :
  | 'isFix'
  | 'isTraversable'
  | 'fov'                    // attributes of "camera"
  | 'type'
  | 'active'
  | 'name'                   // attributes of "character" :
  | 'description'
  | 'life'
  | 'lifeMax'
  | 'lifeMin'   
  | 'nbOfLives'   
  | 'magic'
  | 'magicMax'
  | 'magicMin'
  | 'level'
  | 'attack'
  | 'defense'
  | 'jumpForce'
  | 'maxJumpsInTheAir'
  | 'money'
  | 'class'
  | 'race'
  | 'acceleration'    
  | 'speed'                // attributes of "vehicle" :
  | 'maxSpeed'
  | 'minSpeed'
  | 'boost'
  | 'maxBoost'
  | 'nbMunitions'           // attributes of"weapon" :
  | 'nbMunitionsMax'        
  | 'shootPower'
  | 'damages'               //attributes of "projectile"
  | 'value'                // attributes of "bonus" :
  | 'unit'
  | 'objectname'
  | 'attributName'               
  | 'volume'                 //attributes of "media"
  | 'number'              //attributes of "ball"
  | 'moveWithCamera'
  ;
 
attributListeOuObjet :
  'inventory'
  | 'equipedObjects'
  | 'entrances'
  | 'exits'
  | 'damageZone'
  | 'collectors'
  | 'typesCollectors'
  | 'generators'
  | 'typeGenerators'
  | 'breakers'
  | 'typesBreakers'
  | 'teleportables'
  | 'typesTeleportables'
  ;
 
attributTps :
  'boostInterval'
  | 'shootInterval'        //attributes of "weapon" :
  | 'reloadTime'
  ;
 

//////////////////////////// ( new definitions of actions ) ///////////////////
 
definition : 'definition' ident 'means' consequences;
 
consequences :
  consequ (',' consequ)*
  ;
  
consequ :
  siAlors
  | action
  | affectation
  | activCommande
  | appelDef
  | 'victory'
  | 'defeat'
  ;
 
appelDef :
  ident           //ident of a definition of an action (means)
  ;
 
activCommande :
  ('activate' | 'disable') ('commands' | 'mouse' (souris (',' souris)*)? | 'key' clavier (',' clavier)* | 'keyboard' )
  ;
//disable commands              // all the commands
//disable key                   // all the key commands
//disable mouse up, down        // only move up or down with the mouse
 

//////////////////////////// ( Initialization of commands )  /////////////////////////////

commande :
  'command' (ident 'is' actionCommande (',' actionCommande)* | actionCommande)
  ;

actionCommande :
  ('mouse' souris | 'key' clavier) 'for' (ident | actionCommandePressee | actionCommandeMaintenue) // ident : what is defined with means
  ;
 
souris :
        'up' | 'down' | 'left' | 'right' | 'lClick' | 'cClick' | 'rClick' | 'scrollUp' | 'scrollDown'
        ;
 
clavier :
        CHAR | 'up' | 'down' | 'left' | 'right' | 'space' | 'echap' | 'enter'          //CHAR : Z,Q,S,D,...
        ;
 
actionCommandePressee :
  'jump' operation
  | 'pause'
  | 'stop'
  ;
actionCommandeMaintenue :
  'move' ('left' | 'right' | 'forward' | 'backward')
  | 'turn' ('left' | 'right')
  | 'accelerate'
  | 'brake'
  ;
 
 
//////////////////////////// ( Rules of the game + conditions of victory / defeat )  /////////////////////////////

reglesJeu :
  'rule' (ident 'is')? declencheur 'then' consequences ','
  ;
conditions :
  conditionEt ('or' conditionEt)*
  ;
 
conditionEt :
  conditionsNot ('and' conditionsNot)*
  ;
  
conditionsNot :
  'not' cond
  ;

cond :
  '(' conditions ')'
  | etat
  | operation comparaison operation
  ;

etat :
  accesClasse 'is' ('not')? ('dead' | 'alive' | 'effaced' | 'generated' | 'touching' (('other')? accesGlobal | accesLocal))  // for an object
  | (ident | 'game') 'is' ('not')? ('finished' |'started' | 'paused' | 'muted' ('on' | 'off') | 'played' | 'stopped' )  // game,counter,media
  | 'true'                                                   
  | 'victory'
  | 'defeat'
  ;
 
declencheur :
  accesClasse ('dies' | ('touches' | 'kills') (('other')? accesGlobal | accesLocal) | ('killed' | 'touched') ('by' (('other')? accesGlobal | accesLocal))? )
  | (ident | 'game') ('ends' |'starts')          //ident if it is a counter
  | variable 'becomes' varOuNb
  | ident 'becomes' ('player' | interaction)
  ;
  
siAlors :
  'if' conditions 'then' consequences ('else' consequences)? 'end'
  ;
  
 
action :
  accesClasse actionObjet
  | (ident | 'game') ('ends' |'starts')
  | ('pause' | 'mute' ('on' | 'off') | 'play' | 'stop' ) ident
  | 'block' transformation 'of' accesClasse coordonnees
  | ('efface' | 'generate') (accesClasse | operation accesClasse ('in' accesLocal | 'on' accesLocal | 'at' coordonnees)?)
  | 'wait' operation uniteTps 'then' consequences 'endWait'
  | 'save'
  ;
 
actionObjet :
  'dies'
  | actionCommandePressee
  | actionCommandeMaintenue ('during' operation uniteTps | 'until' conditions)
  | 'equip' (accesLocal | 'next' | 'previous')   
  ;
 
affectation :
  (('assign' | 'add' | 'sub') operation) 'for' variable | 'invert' variable 'with' variable
  ;

//add : a += b, remove : a -= b, assign : a = b, invert : tmp=a; a=b; b=tmp;

coordonnees :
  operation operation operation
  ;
 
comparaison :
  '=' | '<' | '>' | '<=' | '>='
  ; 
 
transformation :
  'translation'
  | 'rotation'
  | 'scale'
  ;
 
uniteTps :
  'mn'
  | 'sec'
  | 'ms'
  | 'frames'
  ;
 
operation :
  ('random' 'between' operationPlus 'and')? operationPlus
  ;
 
operationPlus :
  operationMul (operateurPlus operationMul)*
  ;
operationMul :
  operationPuiss (operateurMul operationPuiss)*
  ;
  
operationPuiss :
  operationparentheses ('^' operationparentheses)*
  ;
  
operationparentheses :
  '(' operation ')'
  | varOuNb
  ;
 
varOuNb :
  variable
  | NUMBER
  ;
 
variable :
  (('x' | 'y' | 'z') 'of' (typeCoordonnees | ident | attribut)) 'of' accesClasse
                      //x of size of num 10 in listeWeapon
  ;
 
accesClasse : accesLocal | accesGlobal;
 
accesGlobal :
  typeObjet
  | interaction
  | 'not' (typeObjet | interaction | 'player')
  | 'all'
  ;
 
accesLocal :
  ident
  | 'num' operation 'in' ident
  | 'player'
  ;
 
 
typeCoordonnees :
  'positition' | 'rotation' | 'scale'
  ;
 
operateur :
  operateurMul
  | operateurPlus
  ;
  
operateurMul :
  '*'
  | '/'
  | '%'
  ;
  
operateurPlus :
  '+'
  | '-'
  ;
 
ident :
  STRING
  ;
 
//////////////////////////// ia //////////////////////////
 
iaBasique : 'ia' accesClasse 'is' actionObjet (',' actionObjet)*;
 

/*------------------------------------------------------------------
* LEXER RULES
*------------------------------------------------------------------*/
BOOL        : ('true' | 'false');
STRING        :  ('a'..'z' | 'A'..'Z') ('a'..'z' | 'A'..'Z' | '0'..'9')+ ;
CHAR   : 'a'..'z' | 'A'..'Z';
NUMBER : ('0'..'9')+ (',' ('0'..'9')+ )? ;
WS  :   ( ' '  
           | '\t'  
           | '\r'  
           | '\n'  
           | '\u000C'
           )+ {$channel=HIDDEN;}  
        ;
\end{lstlisting}


\section{Grammaire bas-niveau}
\label{sec:basniveau}
Cette partie présente la totalité de la grammaire bas-niveau

\begin{lstlisting}[language=Grammar]
 Game ::= refreshLoop eventsManager ressourcesSets camera entities physicsEngine

refreshLoop ::= signalUpdateCounter keyListener{ keyboardCommands } mouseListener{ mouseCommands }
//signalUpdateCounter is triggered by the refreshLoop and updates counters of type times

// pressing a key or action with mouse triggers an event
keyboardCommands ::= (keystroke : signalSets)*

mouseCommands ::= (typeOfClick : signalSets)*

signalSets ::= signal (| signal)*

----------------------------------------------------------------------------------------------------------
// Events Managers

eventsManager ::= signal (@ signal)* instructions (| signal instructions)*

instructions ::= resourceApply(applyExpression) 
               | if conditonnal then instructions (& instructions)* (else instructions (& instructions)*)? 
               | conceptsInstructions

conceptsInstructions ::= gamOver | pause | newGame | saveGame

conditionnal ::= testExpression (booleanOperator testExpression)*

booleanOperator ::= and | or

testExpression ::= expression comparisonOperator expression

comparisonOperator ::= < | > | <= | >= | == | !=

applyExpression ::= arithmeticOperator (metaExpression | expression)

// to reuse an expression without writing it
metaExpression ::= nameExp expression

nameExp ::= string
expression ::= (arithmeticOperator value)+
arithmeticOperator ::= + | - | * | / | %


//random(value,value) to generate a random number between these integers
value = ressource | constant | random(value, value) | random(0, value)
constant ::= int | double

----------------------------------------------------------------------------------------------------------
// Resource Manager

// An event tiggers when a resource is modified
resourcesSets ::= (enumResource | resource)+


enumResource ::= name { nameEnumResource (, nameEnumResource)*}
nameEnumResource ::=  String

resource ::= (# nameEnumResource)? name (signal (@ signal)*)? (timer|initValue)

name ::= string

timer ::= step initTimer

initValue ::= int | double

----------------------------------------------------------------------------------------------------------
// camera and entites manager


camera ::= name position

position ::= vector | angle

entities ::= map with object+

// With a map like a points matrix with a unique texture
map ::= matrix texture


object ::= object = (name object3D parameters | media)
parameters ::= coeffOfFriction = double weight = double speed = vector position = vector  isFixed = boolean isTraversable = boolean
object3D ::= colladaFile boundingBox
media ::= soundLevel | isMute | isPlayed | isStopped | isPaused
colladaFile ::= [a-zA-Z]*.dae

----------------------------------------------------------------------------------------------------------
// physics engine

physicsEngine ::= forces+ collision
forces ::= gravity | wind |...

// Bounded box are provided by the user
collision ::= collision{ name name signalSets (, name name signalSets)*}
gravity ::= gravity = vector
wind ::= wind = vector


\end{lstlisting}


\section{Diagrammes UML des mini-jeux}
\label{sec:uml}
\clearpage

\begin{figure}[h]
 \centering
 \includegraphics[width=\textwidth]{../umls/UML_images/Pacman/class} \hfill
 \caption{Diagramme de classe de Pacman}
\end{figure}

\begin{figure}[h]
 \centering
 \includegraphics[width=10cm]{../umls/UML_images/Pacman/utilisation}
 \caption{Diagramme d'utilisation de Pacman}
\end{figure}

\begin{figure}[h]
 \centering
 \includegraphics[width=9cm]{../umls/UML_images/Pacman/sequence} \hfill
 \includegraphics[width=6cm]{../umls/UML_images/Pacman/sequence2} \hfill
 \caption{Diagrammes de séquence de Pacman}
\end{figure}



\clearpage

\begin{figure}[h]
 \centering
 \includegraphics[height=6cm]{../umls/UML_images/Bat42/utilisation} \hfill
 \includegraphics[height=13cm]{../umls/UML_images/Bat42/class} \hfill
 \caption{En haut, cas d'utilisation de 1942 ; en bas, diagramme de classe de 1942}
\end{figure}

\begin{figure}[h]
 \centering
 \includegraphics[height=8cm]{../umls/UML_images/Bat42/sequence} \hfill
 \includegraphics[width=\textwidth]{../umls/UML_images/Bat42/sequenceIA} \hfill
 \caption{Diagrammes de séquences de 1942}
\end{figure}

% \titlegame{Volley}
%  \uml{Volley/utilisation}{Diagramme de cas d'utilisation du jeu de volley}
%  \uml{Volley/class}{Diagramme de classe du jeu de volley}
%  \uml{Volley/sequence}{Diagramme de séquence du jeu de volley}
%  \uml{Volley/sequence2}{Diagramme de séquence du jeu de volley}
%  \uml{Volley/sequence3}{Diagramme de séquence du jeu de volley}
%  \uml{Volley/sequence4}{Diagramme de séquence du jeu de volley}
% 

\clearpage

\begin{figure}[h]
 \centering
 \includegraphics[width=\textwidth,height=9cm]{../umls/UML_images/course/activity} \hfill
 \includegraphics[width=\textwidth]{../umls/UML_images/course/class} \hfill
 \caption{En haut, diagramme d'activité du jeu de course ; en bas, son diagramme de classes}
\end{figure}

\begin{figure}[h]
 \centering
 \includegraphics[width=\textwidth]{../umls/UML_images/course/sequence} \hfill
 \caption{Diagramme de séquence du jeu de course}
\end{figure}

\clearpage

\begin{figure}[h]
 \centering
 \includegraphics[width=\textwidth]{../umls/UML_images/Mario/Utilisation} \hfill
 \caption{Cas d'utilisation de Mario}
\end{figure}

\begin{figure}[h]
 \centering
 \includegraphics[height=9cm]{../umls/UML_images/Mario/Class} \hfill
 \caption{Diagramme de classes de Mario}
\end{figure}

\begin{figure}[h]
 \centering
 \includegraphics[height=11.5cm]{../umls/UML_images/Mario/Sequence} \hfill
 \caption{Diagramme de séquence de Mario}
\end{figure}


% 
% \titlegame{Watch'N'Droid}
%  \uml{WatchNDroid/class}{Diagramme de classe du jeu Watch'N'Droid}
%  \uml{WatchNDroid/sequence}{Diagramme de séquence du jeu Watch'N'Droid}
% 
\clearpage

\begin{figure}[h]
 \centering
 \includegraphics[height=7cm]{../umls/UML_images/Billard/utilisation} \hfill
 \includegraphics[width=\textwidth,height=12cm]{../umls/UML_images/Billard/class} \hfill
 \caption{En haut, les cas d'utilisation du billard ; en bas, son diagramme de classes}
\end{figure}

\begin{figure}[h]
 \centering
 \includegraphics[height=6.5cm]{../umls/UML_images/Billard/sequence1} \hfill
 \includegraphics[height=11cm]{../umls/UML_images/Billard/sequence2} \hfill
 \caption{Diagrammes de séquence du billard}
\end{figure}

\clearpage

\begin{figure}[h]
 \centering
 \includegraphics[width=\textwidth]{../umls/UML_images/Commissariat/utilisation} \hfill
 \caption{Cas d'utilisation du jeu de gestion}
\end{figure}

\begin{figure}[h]
 \centering
 \includegraphics[width=\textwidth]{../umls/UML_images/Commissariat/class} \hfill
 \caption{Diagramme de classes du jeu de gestion}
\end{figure}

\end{document}
